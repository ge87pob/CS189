

\subsection{Activation Function Implementations:}

Implementation of \texttt{activations.ReLU}:

\begin{lstlisting}[language=Python]
class ReLU(Activation):
    def __init__(self):
        super().__init__()

    def forward(self, Z: np.ndarray) -> np.ndarray:
        """Forward pass for relu activation:
        f(z) = z if z >= 0
               0 otherwise
        
        Parameters
        ----------
        Z  input pre-activations (any shape)

        Returns
        -------
        f(z) as described above applied elementwise to `Z`
        """
        ### YOUR CODE HERE ###
        return ...

    def backward(self, Z: np.ndarray, dY: np.ndarray) -> np.ndarray:
        """Backward pass for relu activation.
        
        Parameters
        ----------
        Z   input to `forward` method
        dY  gradient of loss w.r.t. the output of this layer
            same shape as `Z`

        Returns
        -------
        gradient of loss w.r.t. input of this layer
        """
        ### YOUR CODE HERE ###
        return ...

\end{lstlisting}

Implementation of \texttt{activations.SoftMax}:

\begin{lstlisting}[language=Python]
class SoftMax(Activation):
    def __init__(self):
        super().__init__()

    def forward(self, Z: np.ndarray) -> np.ndarray:
        """Forward pass for softmax activation.
        Hint: The naive implementation might not be numerically stable.
        
        Parameters
        ----------
        Z  input pre-activations (any shape)

        Returns
        -------
        f(z) as described above applied elementwise to `Z`
        """
        ### YOUR CODE HERE ###
        return ...

    def backward(self, Z: np.ndarray, dY: np.ndarray) -> np.ndarray:
        """Backward pass for softmax activation.
        
        Parameters
        ----------
        Z   input to `forward` method
        dY  gradient of loss w.r.t. the output of this layer
            same shape as `Z`

        Returns
        -------
        gradient of loss w.r.t. input of this layer
        """
        ### YOUR CODE HERE ###
        return ...

\end{lstlisting}


\subsection{Layer Implementations:}

Implementation of \texttt{layers.FullyConnected}:

\begin{lstlisting}[language=Python]
class FullyConnected(Layer):
    """A fully-connected layer multiplies its input by a weight matrix, adds
    a bias, and then applies an activation function.
    """

    def __init__(
        self, n_out: int, activation: str, weight_init="xavier_uniform"
    ) -> None:

        super().__init__()
        self.n_in = None
        self.n_out = n_out
        self.activation = initialize_activation(activation)

        # instantiate the weight initializer
        self.init_weights = initialize_weights(weight_init, activation=activation)

    def _init_parameters(self, X_shape: Tuple[int, int]) -> None:
        """Initialize all layer parameters (weights, biases)."""
        self.n_in = X_shape[1]

        ### BEGIN YOUR CODE ###

        W = self.init_weights(...)
        b = ...

        self.parameters = OrderedDict({"W": W, "b": b}) # DO NOT CHANGE THE KEYS
        self.cache: OrderedDict = ...  # cache for backprop
        self.gradients: OrderedDict = ...  # parameter gradients initialized to zero
                                           # MUST HAVE THE SAME KEYS AS `self.parameters`

        ### END YOUR CODE ###

    def forward(self, X: np.ndarray) -> np.ndarray:
        """Forward pass: multiply by a weight matrix, add a bias, apply activation.
        Also, store all necessary intermediate results in the `cache` dictionary
        to be able to compute the backward pass.

        Parameters
        ----------
        X  input matrix of shape (batch_size, input_dim)

        Returns
        -------
        a matrix of shape (batch_size, output_dim)
        """
        # initialize layer parameters if they have not been initialized
        if self.n_in is None:
            self._init_parameters(X.shape)

        ### BEGIN YOUR CODE ###
        
        # perform an affine transformation and activation
        out = ...
        
        # store information necessary for backprop in `self.cache`

        ### END YOUR CODE ###

        return out

    def backward(self, dLdY: np.ndarray) -> np.ndarray:
        """Backward pass for fully connected layer.
        Compute the gradients of the loss with respect to:
            1. the weights of this layer (mutate the `gradients` dictionary)
            2. the bias of this layer (mutate the `gradients` dictionary)
            3. the input of this layer (return this)

        Parameters
        ----------
        dLdY  gradient of the loss with respect to the output of this layer
              shape (batch_size, output_dim)

        Returns
        -------
        gradient of the loss with respect to the input of this layer
        shape (batch_size, input_dim)
        """
        ### BEGIN YOUR CODE ###
        
        # unpack the cache
        
        # compute the gradients of the loss w.r.t. all parameters as well as the
        # input of the layer

        dX = ...

        # store the gradients in `self.gradients`
        # the gradient for self.parameters["W"] should be stored in
        # self.gradients["W"], etc.

        ### END YOUR CODE ###

        return dX

\end{lstlisting}


\subsection{Loss Function Implementations:}

Implementation of \texttt{losses.CrossEntropy}:

\begin{lstlisting}[language=Python]
class CrossEntropy(Loss):
    """Cross entropy loss function."""

    def __init__(self, name: str) -> None:
        self.name = name

    def __call__(self, Y: np.ndarray, Y_hat: np.ndarray) -> float:
        return self.forward(Y, Y_hat)

    def forward(self, Y: np.ndarray, Y_hat: np.ndarray) -> float:
        """Computes the loss for predictions `Y_hat` given one-hot encoded labels
        `Y`.

        Parameters
        ----------
        Y      one-hot encoded labels of shape (batch_size, num_classes)
        Y_hat  model predictions in range (0, 1) of shape (batch_size, num_classes)

        Returns
        -------
        a single float representing the loss
        """
        ### YOUR CODE HERE ###
        return ...

    def backward(self, Y: np.ndarray, Y_hat: np.ndarray) -> np.ndarray:
        """Backward pass of cross-entropy loss.
        NOTE: This is correct ONLY when the loss function is SoftMax.

        Parameters
        ----------
        Y      one-hot encoded labels of shape (batch_size, num_classes)
        Y_hat  model predictions in range (0, 1) of shape (batch_size, num_classes)

        Returns
        -------
        the gradient of the cross-entropy loss with respect to the vector of
        predictions, `Y_hat`
        """
        ### YOUR CODE HERE ###
        return ...

\end{lstlisting}


\subsection{Model Implementations:}

Implementation of \texttt{models.NeuralNetwork.forward}:

\begin{lstlisting}[language=Python]
    def forward(self, X: np.ndarray) -> np.ndarray:
        """One forward pass through all the layers of the neural network.

        Parameters
        ----------
        X  design matrix whose must match the input shape required by the
           first layer

        Returns
        -------
        forward pass output, matches the shape of the output of the last layer
        """
        ### YOUR CODE HERE ###
        # Iterate through the network's layers.
        return ...

\end{lstlisting}

Implementation of \texttt{models.NeuralNetwork.backward}:

\begin{lstlisting}[language=Python]
    def backward(self, target: np.ndarray, out: np.ndarray) -> float:
        """One backward pass through all the layers of the neural network.
        During this phase we calculate the gradients of the loss with respect to
        each of the parameters of the entire neural network. Most of the heavy
        lifting is done by the `backward` methods of the layers, so this method
        should be relatively simple. Also make sure to compute the loss in this
        method and NOT in `self.forward`.

        Note: Both input arrays have the same shape.

        Parameters
        ----------
        target  the targets we are trying to fit to (e.g., training labels)
        out     the predictions of the model on training data

        Returns
        -------
        the loss of the model given the training inputs and targets
        """
        ### YOUR CODE HERE ###
        # Compute the loss.
        # Backpropagate through the network's layers.
        return ...

\end{lstlisting}

Implementation of \texttt{models.NeuralNetwork.predict}:

\begin{lstlisting}[language=Python]
    def predict(self, X: np.ndarray, Y: np.ndarray) -> Tuple[np.ndarray, float]:
        """Make a forward and backward pass to calculate the predictions and
        loss of the neural network on the given data.

        Parameters
        ----------
        X  input features
        Y  targets (same length as `X`)

        Returns
        -------
        a tuple of the prediction and loss
        """
        ### YOUR CODE HERE ###
        # Do a forward pass. Maybe use a function you already wrote?
        # Get the loss. Remember that the `backward` function returns the loss.
        return ...

\end{lstlisting}

